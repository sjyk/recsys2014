\section{Bias Mitigation}
\label{changemod}
In our learning phase, we collect rating triplets ($g_i[j]$, $m[j]$, $g_f[j]$), and in our analysis phase, we determine whether the triplets exhibit statistically significant social influence bias. 
In the mitigation phase, we propose two models: correction model (infers the initial rating given a final rating and the median), and a prediction model (predicts final ratings given an initial rating and the median).
Once trained, the correction model can be applied to correct final grades collected without the triplet (either historical or post-learning).
The prediction model can be used to analyze properties of the social influence bias eg. are ratings above the median affected the same way as ratings below the median.

Previous work, suggests that social influence is not a homogeneous bias, namely, positive influences are different from negative influences.
In Muchnik et al. \cite{muchnik2013social}, they found that when they positively treated posts with higher up-vote counts it lead to a significant increase in the likelihood of additional up votes (32\% more likely). 
On the other hand, they argue negative treatments inspired correction behavior; where some participants wanted to correct what they felt was an incorrect score. 
They found that this also increased the likelihood of up-voting (88\% more likely); as opposed to the conforming response which would be increased down-votes.

These results suggest that the effects of viewing median ratings can be non-linear and are very context/question dependent.
Similar to the previous section where we applied non-parametric tests that did not make a strong assumption about the distribution of the data, we propose a information theoretic polynomial function search that does not make strong assumptions about the nature of the relationship.

\subsection{Correction Model}
Recall that $g_f[j]$ is the final rating for participant j, and $m[j] - g_i[j]$ is the difference between the median and the initial rating.
We want to find a polynomial function $f$ such that:
 \begin{equation} f(g_f[j]) 
 \approx m[j] - g_i[j]
 \end{equation}
Let $f\in \mathcal{P}^k$ be a polynomial of degree $k$.
The square loss of $f$, is the error in predicting $m[j] - g_i[j]$ from $f(g_f[j])$:
\begin{equation}
\mathcal{L}(X_c;f,k) = \sum_j ((m[j] - g_i[j]) - f(g_f[j]))^2 
\end{equation}
For a given $k$, the best-fit polynomial minimizes this square-loss:
\begin{equation}
f^*_k =\arg \min_f \mathcal{L}(X_c;f,k)
\end{equation}
For a given $k$, this problem can be solved with least squares.
To search over the space of polynomial models, we apply a well-studied technique called the Bayesian Information Criterion (BIC) \cite{schwarz1978estimating,burnham2002model}.
This technique converts the optimization problem into a penalized problem that jointly optimizes over the ``complexity parameter" $k$.
This penalty can be interpreted as bias towards lower degree models, in other words, an Occam's Razor prior belief. 
Cross-validation is an alternate method to empirically determine optimal model, and in practice, they give very similar results.
BIC, however, is derived through maximum likelihood estimate and is not an empirical estimate so the learned model has a notion of optimality conditioned on the BIC prior belief.

Thus, we reformulate the optimization problem in the following way to incorporate the BIC penalty:
\begin{equation}
\arg \min_{f,k} |X_c|\log(\mathcal{L}(X_c;f,k)) + k\log(|X_c|)
\end{equation}
The resulting optimal polynomial will tell how to correct a final rating to infer the initial one.
Let q:
\begin{equation}q(j) = m[j] - f(g_f[j])\end{equation}
the predicted initial grade, and this value can be the input to our recommendation algorithm.

\subsection{Applying the Corrections}
There are two ways in which we can apply the correction model to existing recommender systems data.
First, we can train our correction on all triplets, including ones that did not change, to get a correction that we can then apply to all ratings in the post-learning phase.
The second way is to estimate the probability that a rating is changed, and if that probability is above a threshold $\alpha$ (eg. 50\%) we can apply the correction.
With the second way, the correction model is only trained on those triplets where the initial rating is different from the final one.
To estimate this probability, we can apply a logistic regression model to predict whether or not a rating has been changed from all other ratings.
Let $c(i,j)$ be 1 if participant j changed his or her rating for issue i and 0 if not. 
Our feature vector is the vector of all final ratings for that participant $v[j]_f = [g_f^1[j],...,g_f^6[j]]$.
Then, we can apply this logistic regression model to estimate the probability that $c(i,j) = 1$, using the logistic function:
 \begin{equation}
 P[c(i,j) = 1] = \frac{1}{e^{-\beta^Tv[j]_f}}
 \end{equation}
We include results from both approaches in our experiments.

\subsection{Prediction Model}
For the prediction model, we make the dependent variable $m[j] - g_i[j]$ and the independent variable $g_f[j] - g_i[j]$.
We apply the polynomial regression with the BIC optimization as before, and find an optimal function $f$ such that \begin{equation}f(m[j] - g_i[j]) \approx g_f[j] - g_i[j]\end{equation}
$f$ is a function of the difference between the initial rating and the median, that predicts the change in rating.
This model allows us to reason about the nature of the social influence bias in the system.
For example, if $|f(x)| > |f(-x)|$ for $x > 0$, we know that ratings above the median lead to a larger rating change.
Additionally, $f'(x)$ tells us how the change varies as the observed difference with the median increases. 
