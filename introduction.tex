\section{Introduction}
User-assigned ratings are a key component of almost all recommender systems.
Rating data, like traditional surveys, are subject to a variety of biasing tendencies \cite{groves2013survey}.
We explore a class of biases, collectively called \emph{Social Influence Bias}, which arise from feedback from the actions of other participants in the system \cite{demarzo2003persuasion, moscovici1972social, wood2000attitude}.
One aspect of Social Influence Bias is the phenomenon called \emph{Social Herding}, where the feedback from the community encourages future participants to conform to what they perceive as the ``norm" in the community. 
\begin{figure}[t]
  \centering
    \includegraphics[scale=0.30]{../plots/intro.png}
      \caption{In many rating systems, participants see aggregate statistics before leaving a rating. The ratings can be affected by Social Influence Bias, where a participant's action is affected by the aggregate statistic they see. We reveal a median rating after a participant has submitted, and test the hypothesis that participants will subsequently regress towards the median, a process called \emph{Social Herding}.}
      \label{grading-0}
\end{figure}
In a rating system, this can lead to an increased tendency to leave ratings close to the mean or the median rating. 
The effects of social herding are crucial to the design of recommendation algorithms as many algorithms assume statistical independence between different participants and use the spatial relationships between numerical features representing those participants.

A common feedback mechanism is the use of aggregate statistics, for example, showing the average rating for a product before a participant shares his or her rating (Figure \ref{grading-0}).
In many cases, such as product reviews, it is not practical to hide this information from potential raters.
The use of social content is an established user experience design technique to incentivize participation and increase user engagement with the application \cite{shneiderman1992designing}.
Furthermore, an application of particular interest is online participatory democracy where open aggregate results increase the transparency of the system \cite{albors2008new,o2012transparency,noveck2008wiki}.

In recent related work, Muchnik et al. \cite{muchnik2013social}, used a randomized experiment to determine the magnitude of social herding in up-voting in Reddit.com.
They randomly treated forum posts with extra up-votes and down-votes and measured the treatment effect; concluding that a statistically significant social herding tendency exists.
We study a related effect using a new platform, the California Report Card (CRC), which reveals the median values to participants \emph{after} they assign their own ratings, and then allows them to modify their ratings.
We test whether participants who have already submitted a rating will actually \emph{change} their existing ratings upon learning the median rating for the population and if there is a tendency to ``herd" towards the median grade.
For comparision with the CRC, we ran a reference survey through SurveyMonkey which was given to a random sample of 611 participants from the company's paid pool of California participants.
In this survey, we asked participants to respond to the same questions as the CRC on the same grading scale; but without the feedback of the median grade.
Being a randomized survey, the reference survey has less problems with self-selection biases which may exist in the CRC participants. 

The findings of Muchnik et al. suggest that social herding will be observed in the form of a regression towards the observed median grade during the grade changes.
In this paper, we test the hypothesis of social herding, propose a model for the relationship between observed medians and subsequent grade change, and provide experimental results comparing the data from the CRC to the randomized reference survey.

\subsection{Hypotheses and Contributions}
\noindent \textbf{Null Hypothesis} The null hypothesis is that viewing the median grade does not affect how a participant chooses to change his or her grade, and does not affect any future grades given by the participant.

\noindent \textbf{Social Herding Towards the Median} The social herding hypothesis is that when a participant changes their grade, they have a tendency to change towards the median grade. If this hypothesis is true, the final grades of participants who change their grades will be more tightly concentrated around the median than those from participants who did not change their grades and participants from the reference survey.

\noindent \textbf{Social Herding and Question Order} The question order hypothesis states that disagreement with the median on previous questions leads to a future responses bias towards the median. We hypothesize that as participants grade each issue and see the median grades, their initial grades are comparative closer to the median than participants in the reference survey.

We develop non-parametric testing procedure, based on the Wilcoxon Rank-Sum statistic (also called the Mann-Whitney statistic) \cite{lehmann2006nonparametrics}, to test these hypotheses.
We explore this hypothesis in a dataset where ratings can take on 13 possible vales from (A+ to F), lending itself to a non-parametric approach where we do not have to make assumptions about the distribution of the data.
In addition to the hypothesis testing, we model grade changes with a polynomial regression.
As before, to avoid having to make a strong assumption about the structure of the model, we use an information theoretic model to learn a flexibile degree polynomial.

