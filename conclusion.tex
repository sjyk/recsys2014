\section{Conclusion and Future Work}
These results suggest that social influence bias can be significant in
recommender systems and that this bias can be substantially reduced
with machine learning. 
To apply this methodology to other recommender systems, a key question
for future work is how is how to extend the approach to other recommender systems.
We see an opportunity for this methodology in systems that combine their browsing and rating interfaces.
For example, after selection, ie. users purchase a product, click on a movie, etc. the rating can be hidden.
Once the user is ready to rate the item, which can be significant time after selection, we can reveal the average rating again after they have assigned a rating of their own. 
Then, we can apply our methodology to learn, analyze, and mitigate bias in the recommender systems.

An open question is how to extend this work to large item inventories and how much training data is required in such cases.  
One idea is to cluster/classify items into a small number of representative categories and train a model for each category.  
We believe that selecting an optimal set of items for training in this context may be posed as a submodular maximization problem.  
We are looking at applying this methodology to recommender systems in other domains (eg. movies) with alternative regression methods, such as Gaussian Process Regression and LOESS.
We are also interested in performing more user studies where a false median is presented (as in the Asch experiments) and exploring methods to
optimally classify participants as conformers and non-conformists.  
We would also like to study and quantify the role of social influence on textual data.\\
\scriptsize
\textbf{We would like to thank the UC Berkeley AMPLab and the UC CITRIS Data and Democracy Initiative for supporting this project. We also thank Brandie Nonnecke, Allen Huang, Camille Crittenden, John Scott, Tanja Aitamurto, Daniel Catterson, Matti Nelimarkka, Henry Brady, and Lt. Governor Gavin Newsom for their work on the CRC project. This work is supported in part by NSF CISE Expeditions Award CCF-1139158, LBNL Award 7076018, and DARPA XData Award FA8750-12-2-0331, the Blum Center for Developing Economies and the Development Impact Lab (USAID Cooperative Agreement AID-OAA-A-12-00011), part of the USAID Higher Education Solutions Network, gifts from Amazon Web Services, Google, SAP,  The Thomas and Stacey Siebel Foundation, Apple, Inc., C3Energy, Cisco, Cloudera, EMC, Ericsson, Facebook, GameOnTalis, Guavus, HP, Huawei, Intel, Microsoft, NetApp, Pivotal, Splunk, Virdata, VMware, WANdisco and Yahoo!.}\\




