\section{Future Work}
The methods we proposed have several interesting directions of future interest. 
We want to extend our work to quantify biases in textual data. 
The California Report Card collects textual suggestions from participants in addition to the quantiative assesment results. 
Participants are encouraged to read the responses of others before leaving a suggestion of their own.
We suspect that this may lead to a bias in the topics discussed by participants, and we would like to explore how similar non-parametric models can be extended to textual data.

Another compelling direction is to attempt to parameterize our model.
We will explore whether we can model the grades as a mixture of binomial distributions (a discrete analog of a mixture of gaussians), and try to derive optimal tests and models for this data.
Intuitively, parametrization should lead to increased statistical power and better fitting models; assuming that the data fits the underlying parametrization.

\section{Conclusion}
We proposed non-parametric hypothesis tests and models to evaluate the biasing tendency of visible aggregate statistics in the California Report Card.
We found that revealing the median led to a statistically significantly tighter grouping of grades around the shown median grade.

We modeled the biasing effect as a regression towards the median grade and fit polynomial to represent the functional relationship between a participant's observed difference with the median and then subsequent grade change.
We applied an information theoretic criteria to select a model of appropriate complexity.
We found that this relationship was quadratic in two out of the six issues, representing a heterogeneity in biasing for positive and negative differences with the median.
We further showed how non-parametric ideas could be extended to the problem of Wilcoxon shift parameter estimation and quantify the effects of the biasing tendency.

In principle, the methods we proposed can be applied to test and model biases in a wide variety input mechanisms.
This is a key motivation for our non-parametric approach.
Understanding these biases, can give insight into the behavior of recommender systems that train on such data.
