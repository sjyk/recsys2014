\section{Related Work}

%% In the context of recommender systems, social influence has been studied primarily in order to use information about social and trusted friend networks to improve recommendations. Jamali et al. described a stochastic block model which predicts recommendations based on both social relations and rating behavior [A]. Shang et al. described models for for improving recommendations among individuals using the theory of social contagion, and among groups using network theory [B]. Ye et al. proposed a quantification metric of social influence, and proposed probabilstic model to model the decision of item selection [C].

%% However, the Asch model for confirmity suggest a particular biasing effect from an aggregate or ``crowd'' (i.e. the previous raters are anonymous to the current rater). This phenomenon known as social herding ...

The Asch model for conformity is the theoretical basis for \emph{social herding}, the tendency to conform \cite{banerjee1992simple,bikhchandani2000herd}, and herding has been a popular consumer choice model in economics \cite{burnkrant1975informational,dholakia2002auction,huang2006herding}. 
Such models have also been studied in psychology as ``persuasion bias" \cite{demarzo2003persuasion}.
In 2011, Lorenz et al. described how these biases can undermine the effectiveness of crowd intelligence in estimation tasks \cite{lorenz2011social}. 
They argue that social herding causes a diminished diversity of opinion potentially leading to inefficiencies and inaccurate collective estimates.
Danescu-Niculescu-Mizil et al. analyze helpfulness ratings on Amazon product reviews \cite{danescu2009opinions}.
They found that the helpfulness ratings did not just depend on the content of the review but also its aggregate score and its relationship to other scores.
In order to better distinguish social influence from other biases, Muchnik et. al designed a randomized experiment in which comments in an online forum were randomly up-treated or down-treated \cite{muchnik2013social}.
They concluded a statistically significant bias where a positive treatment increased the likelihood of positive ratings by 32\%. 
In both Danescu-Niculescu-Mizil et al. and Muchnik et al., they looked at the problem of Social Influence bias in an a priori setting, where users see the aggregate statistic before giving their rating.
Our work tests for a particular form of social influence where users are given the opportunity to change their opinions following the feedback. 

Zhu et al. conducted an experiment in which users evaluate an image on a subjective question with binary scale (eg. ``Is this image cute?"), which was followed (either immediately or later) by a presentation of the crowd consensus opinion \cite{zhu2012switch}. 
Users were given an opportunity to change their response, and they concluded that there was a significant tendency to change submissions.
The tendency to change was the strongest when users were asked to make their second decision much later and not immediately after the first.
However, Zhu et al. also acknowledge there are competing psychological factors at work in this experiment.
Along these lines, Sipos et al. argue that context along with an aggregate rating plays a large role in the users' ratings. That is, users may attempt to ``correct" the average, by voting in a more polarizing manner (more positively or negatively) \cite{siposreview}.
We extend this prior work to measure and predict these changes when the input is more complex than a binary scale, and propose a non-parametric methodology that can be, in principle, extended to a variety of different input mechanisms.
Our model can also account for a changing aggregate statistic such as a median rating changing as more data is collected. 


%% [A] M. Jamali, T. Huang, and M. Ester, ``A Generalized Stochastic Block Model for Recommendation in Social Rating Networks'', in ACM Conference in Recommender Systems (RecSys'11) , Chicago, IL, USA, October 2011.

%% [B] Shang, Shang, et al. ``Wisdom of the crowd: Incorporating social influence in recommendation models.'' Parallel and Distributed Systems (ICPADS), 2011 IEEE 17th International Conference on. IEEE, 2011.

%% [C] Ye, Mao, Xingjie Liu, and Wang-Chien Lee. ``Exploring social influence for recommendation: a generative model approach.'' Proceedings of the 35th international ACM SIGIR conference on Research and development in information retrieval. ACM, 2012.



