\section{Related Work}
In Asch's famous conformity experiments \cite{asch1956studies, asch1955opinions, bond1996culture}, groups of participants were asked to match a line with a set of three different sized lines one of which was of the correct size.
In reality, only one of the participants was ``real" and the others were actors who unanimously chose an incorrect choice.
On average, 25\% of participants conformed to the incorrect consensus compared to 1\% of incorrect answers in a control group.

The Asch model for conformity is the theoretical basis for social herding \cite{banerjee1992simple,bikhchandani2000herd}, and herding has been a popular consumer choice model in economics \cite{burnkrant1975informational,dholakia2002auction,huang2006herding}. 
In 2011, Lorenz et al. described how these biases can undermine the effectiveness of crowd intelligence in estimation tasks \cite{lorenz2011social}. 
They argue that social herding causes a diminished diversity of opinion potentially leading to inefficiencies and inaccurate collective estimates.
Danescu-Niculescu-Mizil et al. analyze helpfulness ratings on Amazon product reviews \cite{danescu2009opinions}.
They found that the helpfulness ratings did not just depend on the content of the review but also its aggregate score and its relationship to other scores.

Such models have been studied in psychology as ``persuasion bias" \cite{demarzo2003persuasion}.
In order to better distinguish social influence from other biases, Muchnik et. al designed a randomized experiment in which comments on Reddit.com were randomly up-treated or down-treated \cite{muchnik2013social}.
They concluded a statistically significant bias where a positive treatment increased the likelihood of positive ratings by 32\%. 

This work tests for a particular form of social influence where users are given the opportunity to change their opinions following feedback. 
Zhu et al. conducted an experiment in which users evaluate an image on a subjective question with binary scale (eg. ``Is this image cute?"), which was followed (either immediately or later) by a presentation of the crowd consensus opinion \cite{zhu2012switch}. 
Users were given an opportunity to change their response, and they concluded that there was a significant tendency to change submissions.
The tendency to change was the strongest when users were asked to make their second decision much later and not immediately after the first.
However, Zhu et al. also acknowlege there are competing psychological factors at work in this experiment.
Along these lines, Sipos et al. argue that context along with an aggregate rating plays a large role in the users' ratings. That is, users may attempt to ``correct" the average, by voting in a more polarizing manner (more positively or negatively) \cite{siposreview}.




