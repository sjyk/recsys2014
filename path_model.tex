\section{Social Herding and Question Order}
\label{path}
In this section, we develop a model for testing the effect of the sequence of ratings.
Order effects have been well studied in surveying \cite{krosnick1987evaluation}, and we look at order effects in the context of social herding. 
Recall, that we posed the each of the six questions in a fixed order.
Our question of interest is: given a participant's average disagreement with the median grade (measured by the absolute deviation) on the previous issues, how is their grade to the following issue affected?
We hypothesize that participants will become more moderate in their grades if they observe that their grades a consistently in disagreement with the population consensus.

This hypothesis is challenging to test as responses to issues may be correlated; even excluding any form of bias.
Consider the following example, if the grades are positively correlated, then low grades on one question could imply even lower grades on another.
In this case, we would see an increase in deviations even though it is not attributable to the biasing tendency.
Consequently, we build a model that compares the CRC to the SurveyMonkey reference survey.
We test to see if the relationship between the deviation of a participant's past grades and their current grades is different between the CRC and reference survey.

Let $d_{kj}$ be the absolute deviation from the median grade of participant $j$'s grade on issue $k$. 
We define a statistic $\bar{d}_{kj}$, which is the mean of all of the absolute deviations on the previous issues:
\begin{equation}
\bar{d}_{kj} = \frac{1}{k-1} \sum_{l < k}  d_{lj}
\end{equation}
If an issue was skipped by participant $j$, we exclude it from the average.
For each issue $k > 1$, we can get a set of differences between the absolute deviation of the current issue and $\bar{d}_{kj}$:
\begin{equation}
D_k = \{(\bar{d}_{kj}-d_{kj})\} \forall j
\end{equation}
We can calculate the same set of deviations for the reference survey which we call $D_k^{(r)}$.
When the differences in $D_k$ are on average positive it means that on issue $k$ participants were more moderate than previous issues and vice versa if the differences are negative.
So formally, our hypothesis test compares whether the set of differences in the CRC $D_k$ are larger than the set of differences in the reference survey $D_k^{(r)}$.
A significant result means that in comparison to the reference survey, CRC participants showed a greater tendency to center their grades around the median after disagreeing on previous issues.

We can apply the same Wilcoxon rank-sum model discussed in the previous section to test this hypothesis.
The testing procedure is the following: (1) we rank the differences in $D_k \cup D_k^{(r)}$, (2) we calculate $W$ which is the sum of the ranks in $D_k$, and 
(3) using the equation from the previous section we test the calculated W under the null hypothesis distribution.
The null distribution models the null hypothesis that there is no difference between $D_k$ and $D_k^{(r)}$, and given this hypothesis what is the probability we will observe the rank-sum statistic $W$.

This test is particularly interesting in the context of initial grades.
If we construct our set $D_k$ so that $\bar{d}_{kj}$ is based on final grades and $d_{kj}$ is the deviation of the initial grade, we can test to see how the concentration of grades around the median changes even without the biasing effect of revealing the median.
The implications of this question are interesting since this tests whether participants have a tendency to \emph{guess} the median grade after prior disagreement with the median.
